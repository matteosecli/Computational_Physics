%!TEX encoding = UTF-8 Unicode
%
% Laboratorio di Fisica III
% Esperienza 11
% Anno accademico 2013/2014
% Daniele Brugnara, Alessandro Casalino
%

\documentclass {article}
\usepackage[utf8]{inputenc}
\usepackage{fontenc}
\usepackage[english]{babel}
\usepackage{graphicx}
\usepackage{float}
\usepackage{fancyhdr}
\usepackage{listingsutf8}
\usepackage{xcolor}
\usepackage{amsfonts}
\usepackage{amsmath}
\usepackage{amssymb}	
\usepackage{wrapfig}
\usepackage{enumitem}
\usepackage{subfigure}
\usepackage [a4paper, top=2.5cm, bottom=2cm, left=1.5cm, right=1.5cm] {geometry}
\pagestyle{fancy}

% cambiato bottom da 1.8 + logo

\makeatletter
\@addtoreset{section}{part}
\makeatother
\rhead{\LARGE Project 1}

\lhead{\large Numerically solving a differential equation}
\lfoot{D. Brugnara, M. Seclì}
\cfoot{}
\rfoot{\thepage}
\renewcommand{\headrulewidth}{0.7pt}
\renewcommand{\footrulewidth}{0.7pt}

\definecolor{dkgreen}{rgb}{0,0.6,0}
\definecolor{dred}{rgb}{0.545,0,0}
\definecolor{dblue}{rgb}{0,0,0.545}
\definecolor{lgrey}{rgb}{0.9,0.9,0.9}
\definecolor{gray}{rgb}{0.4,0.4,0.4}
\definecolor{darkblue}{rgb}{0.0,0.0,0.6}
\lstdefinelanguage{cpp}{
      backgroundcolor=\color{lgrey},  
      basicstyle=\footnotesize \ttfamily \color{black} \bfseries,   
      breakatwhitespace=false,       
      breaklines=true,               
      captionpos=b,                   
      commentstyle=\color{dkgreen},   
      deletekeywords={...},          
      escapeinside={\%*}{*)},                  
      frame=single,                  
      language=C++,                
      keywordstyle=\color{purple},  
      morekeywords={BRIEFDescriptorConfig,string,TiXmlNode,DetectorDescriptorConfigContainer,istringstream,cerr,exit}, 
      identifierstyle=\color{black},
      stringstyle=\color{blue},      
      numbers=right,                 
      numbersep=5pt,                  
      numberstyle=\tiny\color{black}, 
      rulecolor=\color{black},        
      showspaces=false,               
      showstringspaces=false,        
      showtabs=false,                
      stepnumber=1,                   
      tabsize=5,                     
      title=\lstname,                 
    }

\begin{document}

\section{\LARGE Numerically solving a differential equation through a linear system}

\subsection{Abstract}

In this project we aim to solve a special kind of differential equation using a numerical procedure that allows us to express the equation through a linear system. We will study some algorithms to solve such a problem, focusing on the efficiency of the program, setting our goal more on speed than generality. 

The differential equation we're interested in studying is of the type

\begin{equation}
	u''(x)= - f(x)
	\label{differential_eq}
\end{equation}

In our case we will limit our solutions using the contour conditions of $u(0)=0$ and $u(L)=0$, where $[0, L]$ is our domain of integration.
Using Taylor expansion it is possible to express the second derivative of a function $u(x)$ as

\begin{equation}
	u''(x)= \frac{u(x-h)-2 u(x)+u(x+h)}{h^2}+ \O (h^2)
\end{equation}

We are therefore able to discretize equaition (\ref{differential_eq}) using $N$ points, obtaining:

$$u''_i= \frac{u_{i-1}-2 u_i+u_{i+1}}{h^2}=-f_i \quad \quad i \in \left\lbrace 1 \cdots N\right\rbrace$$

Using the matrix representation, we can write equation (\ref{differential_eq}) as

\begin{equation}
 \begin{pmatrix}
   2 & -1 &  0 & 0 & \cdots & 0  \\
  -1 &  2 & -1 & 0 & \cdots & 0  \\
   0 &-1 &  2 & -1 & \cdots & 0 \\
  \vdots  & \vdots  & & \ddots & & \vdots   \\
   0 &  0 & \cdots  & -1 & 2 & -1 \\
   0 &  0 & \cdots & \cdots  & -1 & 2
 \end{pmatrix}
 \begin{pmatrix}
  u_0 \\
  u_1 \\
  u_2 \\
  \vdots  \\
  u_{N-2} \\
  u_{N-1} 
 \end{pmatrix}
 = h^2
 \begin{pmatrix}
  f_0 \\
  f_1 \\
  f_2 \\
  \vdots  \\
  f_{N-2} \\
  f_{N-1} 
 \end{pmatrix}
\end{equation}

Note how, with this system it is already implied that $f(0)=0$ e $f(L)=0$, since the first and last equations state that

$$h^2 f_0=\frac{2 u_0-u_1}{h^2}=\frac{-1 u_{-1}+2 u_0-u_1}{h^2}$$

$$h^2 f_{N-1}=\frac{-u_{N-2}+2 u_{N-1}}{h^2}=\frac{-u_{N-2}+2 u_{N-1}-u_N}{h^2}$$ 

Since the boundary conditions of the differential equations state that $u_{-1}=u(0)=0$ and $u_{N}=u(L)=0$.  

This linear system is indeed very particular and has a clear pattern. We will first focus on finding a solving algorithm for a general tridiagonal matrix and after we will try to implement another program to solve this particular system with the intent of lowering the number of calculation and therefore the computation time.

\subsection{General algorithm for solving a tridiagonal matrix through back and forward substitution}

A general tridiagonal system can be expressed as

\begin{equation}
 \begin{pmatrix}
   b_0 & c_0 &  0 & 0 & \cdots & 0  \\
  a_1 & b_1 & c_1 & 0 & \cdots & 0  \\
   0 & a_2 &  b_2 & c_2 & \cdots & 0 \\
  \vdots  & \vdots  & & \ddots & & \vdots   \\
   0 &  0 & \cdots  & a_{N-2} & b_{N-2} & c_{N-2} \\
   0 &  0 & \cdots & \cdots  & a_{N-1} & b_{N-1}
 \end{pmatrix}
 \begin{pmatrix}
  u_0 \\
  u_1 \\
  u_2 \\
  \vdots  \\
  u_{N-2} \\
  u_{N-1} 
 \end{pmatrix}
 =h^2
 \begin{pmatrix}
  f_0 \\
  f_1 \\
  f_2 \\
  \vdots  \\
  f_{N-2} \\
  f_{N-1} 
 \end{pmatrix}
\end{equation}

\subsection{Particular algorithm}

Using the regular Gaussian elimination algorithm we proceed to find a specific solution of our system as follows:

\begin{equation}
\left(
\begin{array}{cccccc|c}
   2 & -1 &  0 & 0 & \cdots & 0 & f_0 \\
  -1 &  2 & -1 & 0 & \cdots & 0 & f_1 \\
   0 &-1 &  2 & -1 & \cdots & 0 & f_2\\
  \vdots  & \vdots  & & \ddots & & \vdots & \vdots  \\
   0 &  0 & \cdots  & -1 & 2 & -1 & f_{N-2} \\
   0 &  0 & \cdots & \cdots  & -1 & 2 & f_{N-1} 
\end{array}	
\right)
\longrightarrow
\left(
\begin{array}{cccccc|c}
   2 & -1 &  0 & 0 & \cdots & 0 & f_0 \\
   0 &  3 & -2 & 0 & \cdots & 0 & 2 f_1+f_0 \\
   0 & 0 &  4 & -3 & \cdots & 0 & 3 f_2+2 f_1+f_0\\
  \vdots  & \vdots  & & \ddots & & \vdots & \vdots  \\
   0 &  0 & \cdots  & 0 & N & -(N-1) &  (N-1) f_{N-2} + \sum_{j=0}^{N-3} (j+1)f_{j}  \\
   0 &  0 & \cdots & \cdots  & 0 & N+1 &  \sum_{j=0}^{N-1} (j+1)f_{j} 
\end{array}	
\right)
\end{equation}

We therefore know ahead of times the explicit form of the matrix in upper triangular form and are able to compute all the constant terms of the system as follows

$$\tilde{f}_i= \sum_{j=0}^{i}(j+1) f_j$$

However, we don't need to compute the sum every time, since we can compute $\tilde{f}_i$ knowing $\tilde{f}_{i-1}$:

$$\tilde{f}_i=(i+1)f_i+\tilde{f}_{i-1}$$

Once this forward computations are completed, it is possible to proceed with a back substitution, knowing that

$$u_{N-1}=\frac{1}{N+1} \tilde{f}_{N-1}$$

we are able to find the vector of solutions $u_i$

$$u_i=\frac{1}{i+2} (\tilde{f}[i]+ (i+1) u[i+1])$$

Translating the algorithm in C++ code we obtain the following cycle:

\begin{lstlisting}[language=cpp]
	for (i=1; i!=n; i++) {
		f[i]=(i+1)*f[i]+f[i-1];
	}
	u[n-1]=f[n-1]/(n+1);
	for (i=n-2; i>=0; i--) {
		u[i]=(f[i]+(i+1)*u[i+1])/(i+2);
	}
\end{lstlisting}

Similarly to the previous code of section 1.2 this algorithm is characterized by 2 for cycles and therefore the time required for the solution increase linearly with the number of points used.

Moreover, this program allows us to use a minimal amount of memory, storing only the vector f and the solution u. 

\subsection{Errors}

\end{document}